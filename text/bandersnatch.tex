\section{Bandersnatch VRF}\label{sec:bandersnatch}

The Bandersnatch curve is defined by \cite{cryptoeprint:2021/1152}.

The singly-contextualized Bandersnatch Schnorr-like signatures $\bandersig{k}{c}{m}$ are defined as a formulation under the \emph{IETF} \textsc{vrf} template specified by \cite{hosseini2024bandersnatch} (as IETF VRF) and further detailed by \cite{rfc9381}.

\begin{align}
  \bandersig{k \in \bskey}{c \in \hash}{m \in \blob} \subset \blob_{96} &\equiv \set{ x \mid x \in \blob_{96}, \text{verify}(k, c, m, x) = \top }  \\
  \banderout{s \in \bandersig{k}{c}{m}} \in \hash &\equiv \text{output}(x \mid x \in \bandersig{k}{c}{m})_{\dots32}
\end{align}

The singly-contextualized Bandersnatch Ring\textsc{vrf} proofs $\bandersnatch{r}{c}{m}$ are a zk-\textsc{snark}-enabled analogue utilizing the Pedersen \textsc{vrf}, also defined by \cite{hosseini2024bandersnatch} and further detailed by \cite{cryptoeprint:2023/002}.

\begin{align}
  \getringroot(\seq{\bskey}) \in \ringroot &\equiv \text{commit}(\seq{\bskey})  \\
  \bandersnatch{r \in \ringroot}{c \in \hash}{m \in \blob} \subset \blob_{784} &\equiv \set{ x \mid x \in \blob_{784}, \text{verify}(r, c, m, x) = \top }  \\
  \banderout{p \in \bandersnatch{r}{c}{m}} \in \hash &\equiv \text{output}(x \mid x \in \bandersnatch{r}{c}{m})_{\dots32}
\end{align}

Note that in the case a key $\bskey$ has no corresponding Bandersnatch point when constructing the ring, then the Bandersnatch \emph{padding point} as stated by \cite{hosseini2024bandersnatch} should be substituted.
