\section{Introduction}

\subsection{Nomenclature}

%As its namesake paper pointed out, the name \emph{Polkadot} means several things depending on context. It might mean variously the product vision set forth in said paper of a scalable heterogeneous multichain; the decentralized blockchain network and service which came to be along with its state and history; or possibly the protocol and technology-base utilized in order to maintain and run this network and others such as \emph{Kusama}.

In this paper, we introduce a decentralized, crypto-economic protocol to which the Polkadot Network will transition itself in a major revision on the basis of approval by its governance apparatus.

An early, unrefined, version of this protocol was first proposed in Polkadot Fellowship \textsc{rfc}\oldstylenums{31}, known as \emph{CoreJam}. CoreJam takes its name after the collect/refine/join/accumulate model of computation at the heart of its service proposition. While the CoreJam \textsc{rfc} suggested an incomplete, scope-limited alteration to the Polkadot protocol, \Jam refers to a complete and coherent overall blockchain protocol.

\subsection{Driving Factors}

Within the realm of blockchain and the wider Web3, we are driven by the need first and foremost to deliver resilience. A proper Web3 digital system should honor a declared service profile---and ideally meet even perceived expectations---regardless of the desires, wealth or power of any economic actors including individuals, organizations and, indeed, other Web3 systems. Inevitably this is aspirational, and we must be pragmatic over how perfectly this may really be delivered. Nonetheless, a Web3 system should aim to provide such radically strong guarantees that, for practical purposes, the system may be described as \emph{unstoppable}.

While Bitcoin is, perhaps, the first example of such a system within the economic domain, it was not general purpose in terms of the nature of the service it offered. A rules-based service is only as useful as the generality of the rules which may be conceived and placed within it. Bitcoin's rules allowed for an initial use-case, namely a fixed-issuance token, ownership of which is well-approximated and autonomously enforced through knowledge of a secret, as well as some further elaborations on this theme.

Later, Ethereum would provide a categorically more general-purpose rule set, one which was practically Turing complete.\footnote{The gas mechanism did restrict what programs can execute on it by placing an upper bound on the number of steps which may be executed, but some restriction to avoid infinite-computation must surely be introduced in a permissionless setting.} In the context of Web3 where we are aiming to deliver a massively multiuser application platform, generality is crucial, and thus we take this as a given.

Beyond resilience and generality, things get more interesting, and we must look a little deeper to understand what our driving factors are. For the present purposes, we identify three additional goals:
\begin{enumerate}
  \item \label{enum:resilience} Resilience: highly resistant from being stopped, corrupted and censored.
  \item \label{enum:generality} Generality: able to perform Turing-complete computation.
  \item \label{enum:performance} Performance: able to perform computation quickly and at low cost.
  \item \label{enum:coherency} Coherency: the causal relationship possible between different elements of state and thus how well individual applications may be composed.
  \item \label{enum:accessibility} Accessibility: negligible barriers to innovation; easy, fast, cheap and permissionless.
\end{enumerate}

As a declared Web3 technology, we make an implicit assumption of the first two items. Interestingly, items \ref{enum:performance} and \ref{enum:coherency} are antagonistic according to an information theoretic principle which we are sure must already exist in some form but are nonetheless unaware of a name for it. For argument's sake we shall name it \emph{size-coherency antagonism}.

\subsection{Scaling under Size-Coherency Antagonism}

Size-coherency antagonism is a simple principle implying that as the state-space of information systems grow, then the system necessarily becomes less coherent. It is a direct implication of principle that causality is limited by speed. The maximum speed allowed by physics is $C$ the speed of light in a vacuum, however other information systems may have lower bounds: In biological system this is largely determined by various chemical processes whereas in electronic systems is it determined by the speed of electrons in various substances. Distributed software systems will tend to have much lower bounds still, being dependent on a substrate of software, hardware and packet-switched networks of varying reliability.

The argument goes:
\begin{enumerate}
  \item The more state a system utilizes for its data-processing, the greater the amount of space this state must occupy.
  \item The more space used, then the greater the mean and variance of distances between state-components.
  \item As the mean and variance increase, then time for causal resolution (i.e. all correct implications of an event to be felt) becomes divergent across the system, causing incoherence.
\end{enumerate}

Setting the question of overall security aside for a moment, we can manage incoherence by fragmenting the system into causally-independent subsystems, each of which is small enough to be coherent. In a resource-rich environment, a bacterium may split into two rather than growing to double its size. This pattern is rather a crude means of dealing with incoherency under growth: intra-system processing has low size and total coherence, inter-system processing supports higher overall sizes but without coherence. It is the principle behind meta-networks such as Polkadot, Cosmos and the predominant vision of a scaled Ethereum (all to be discussed in depth shortly). Such systems typically rely on asynchronous and simplistic communication with ``settlement areas'' which provide a small-scoped coherent state-space to manage specific interactions such as a token transfer.

The present work explores a middle-ground in the antagonism, avoiding the persistent fragmentation of state-space of the system as with existing approaches. We do this by introducing a new model of computation which pipelines a highly scalable, \emph{mostly coherent} element to a synchronous, fully coherent element. Asynchrony is not avoided, but we bound it to the length of the pipeline and substitute the crude partitioning we see in scalable systems so far with a form of ``cache affinity'' as it typically seen in multi-\textsc{cpu} systems with a shared \textsc{ram}.

Unlike with \textsc{snark}-based L2-blockchain techniques for scaling, this model draws upon crypto-economic mechanisms and inherits their low-cost and high-performance profiles and averts a bias toward centralization.

\subsection{Document Structure}

We begin with a brief overview of present scaling approaches in blockchain technology in section \ref{sec:previouswork}. In section \ref{sec:notation} we define and clarify the notation from which we will draw for our formalisms.

We follow with a broad overview of the protocol in section \ref{sec:overview} outlining the major areas including the Polka Virtual Machine (\textsc{pvm}), the consensus protocols Safrole and \textsc{Grandpa}, the common clock and build the foundations of the formalism.

We then continue with the full protocol definition split into two parts: firstly the correct on-chain state-transition formula helpful for all nodes wishing to validate the chain state, and secondly, in sections \ref{sec:workpackagesandworkreports} and \ref{sec:bestchain} the honest strategy for the off-chain actions of any actors who wield a validator key.

The main body ends with a discussion over the performance characteristics of the protocol in section \ref{sec:discussion} and finally conclude in section \ref{sec:conclusion}.

The appendix contains various additional material important for the protocol definition including the \textsc{pvm} in appendices \ref{sec:virtualmachine} \& \ref{sec:virtualmachineinvocations}, serialization and Merklization in appendices \ref{sec:serialization} \& \ref{sec:statemerklization} and cryptography in appendices \ref{sec:merklization}, \ref{sec:bandersnatch} \& \ref{sec:erasurecoding}. We finish with an index of terms which includes the values of all simple constant terms used in the work in appendix \ref{sec:definitions}, and close with the bibliography.
