\section{Index of Notation}\label{sec:definitions}

\subsection{Sets}

\subsubsection{Regular Notation}
\begin{description}
  \item[$\N$] The set of non-negative integers. Subscript denotes one greater than the maximum. See section \ref{sec:numbers}.
  \begin{description}
    \item[$\N^+$] The set of positive integers (not including zero).
    \item[$\N_B$] The set of balance values. Equivalent to $\N_{2^{64}}$. See equation \ref{eq:balance}.
    \item[$\N_G$] The set of unsigned gas values. Equivalent to $\mathbb{N}_{2^{64}}$. See equation \ref{eq:gasregentry}.
    \item[$\N_L$] The set of blob length values. Equivalent to $\N_{2^{32}}$. See section \ref{sec:numbers}.
    \item[$\N_R$] The set of register values. Equivalent to $\N_{2^{64}}$. See equation \ref{eq:gasregentry}.
    \item[$\N_S$] The set from which service indices are drawn. Equivalent to $\N_{2^{32}}$. See section \ref{eq:serviceaccounts}.
    \item[$\N_T$] The set of timeslot values. Equivalent to $\N_{2^{32}}$. See equation \ref{eq:time}.
  \end{description}
  \item[$\mathbb{Q}$] The set of rational numbers. Unused.
  \item[$\mathbb{R}$] The set of real numbers. Unused.
  \item[$\mathbb{Z}$] The set of integers. Subscript denotes range. See section \ref{sec:numbers}.
  \begin{description}
    \item[$\Z_G$] The set of signed gas values. Equivalent to $\mathbb{Z}_{-2^{63}\dots2^{63}}$. See equation \ref{eq:gasregentry}.
  \end{description}
\end{description}

\subsubsection{Custom Notation}
\begin{description}
  \item[$\mathbb{A}$] The set of service accounts. See equation \ref{eq:serviceaccount}.
  \item[$\mathbb{B}$] The set of Boolean sequences/bitstrings. Subscript denotes length. See section \ref{sec:sequences}.
  \item[$\mathbb{C}$] The set of seal-key tickets. See equation \ref{eq:ticket}. \emph{Not used as the set of complex numbers.}
  \item[$\mathbb{D}$] The set of dictionaries. See section \ref{sec:dictionaries}.
  \begin{description}
    \item[$\dict{K}{V}$] The set of dictionaries making a partial bijection of domain $K$ to range $V$. See section \ref{sec:dictionaries}.
  \end{description}
  \item[$\mathbb{E}$] The set of valid Ed25519 signatures. A subset of $\Y_{64}$. See section \ref{sec:cryptography}.
  \begin{description}
    \item[$\sig{K}{M}$] The set of valid Ed25519 signatures of the key $K$ and message $M$. A subset of $\mathbb{E}$. See section \ref{sec:cryptography}.
  \end{description}
  \item[$\mathbb{F}$] The set of Bandersnatch signatures. A subset of $\Y_{64}$. See section \ref{sec:cryptography}. \emph{NOTE: Not used as finite fields.}
  \begin{description}
    \item[$\bandersig{K}{C}{M}$] The set of Bandersnatch signatures of the public key $K$, context $C$ and message $M$. A subset of $\mathbb{F}$. See section \ref{sec:cryptography}.
    \item[$\bar{\mathbb{F}}$] The set of Bandersnatch Ring\textsc{vrf} proofs. See section \ref{sec:cryptography}.
    \item[$\bandersnatch{R}{C}{M}$] The set of Bandersnatch Ring\textsc{vrf} proofs of the root $R$, context $C$ and message $M$. A subset of $\bar{\mathbb{F}}$. See section \ref{sec:cryptography}.
  \end{description}
  \item[$\mathbb{G}$] The set of data segments, equivalent to octet sequences of length $\mathsf{W}_G$. See equation \ref{eq:segment}.
  \item[$\H$] The set of 32-octet cryptographic values. A subset of $\Y_{32}$. $\H$ without a subscript generally implies a hash function result. See section \ref{sec:cryptography}. \emph{NOTE: Not used as quaternions.}
  \begin{description}
    \item[$\H_B$] The set of Bandersnatch public keys. A subset of $\Y_{32}$. See section \ref{sec:cryptography} and appendix \ref{sec:bandersnatch}.
    \item[$\H_E$] The set of Ed25519 public keys. A subset of $\Y_{32}$. See section \ref{sec:signing}.
  \end{description}
  \item[$\mathbb{I}$] The set of work items. See equation \ref{eq:workitem}.
  \item[$\mathbb{J}$] The set of work execution errors.
  \item[$\mathbb{K}$] The set of validator key-sets. See equation \ref{eq:validatorkeys}.
  \item[$\mathbb{L}$] The set of work-digests.
  \item[$\mathbb{M}$] The set of \textsc{pvm} \textsc{ram} states. See equation \ref{eq:pvmmemory}.
  \item[$\operandtuple$] The accumulation operand element, corresponding to a single work-item.
  \item[$\mathbb{P}$] The set of work-packages. See equation \ref{eq:workpackage}.
  \item[$\avspec$] The set of availability specifications.
  \item[$\mathbb{T}$] The set of deferred transfers.
  \item[$\mathbb{U}$] The set of partial state, used during accumulation. See equation \ref{eq:partialstate}.  \item[$\mathbb{V}_{\memory}$] The set of validly readable indices for \textsc{pvm} \textsc{ram} $\memory$. See appendix \ref{sec:virtualmachine}.
  \item[$\mathbb{V}^*_{\memory}$] The set of validly writable indices for \textsc{pvm} \textsc{ram} $\memory$. See appendix \ref{sec:virtualmachine}.
  \item[$\mathbb{W}$] The set of work-reports.
  \item[$\mathbb{X}$] The set of refinement contexts.
  \item[$\Y$] The set of octet strings/``blobs''. Subscript denotes length. See section \ref{sec:sequences}.
  \begin{description}
    \item[$\Y_{BLS}$] The set of BLS public keys. A subset of $\Y_{144}$. See section \ref{sec:signing}.
    \item[$\Y_R$] The set of Bandersnatch ring roots. A subset of $\Y_{144}$. See section \ref{sec:cryptography} and appendix \ref{sec:bandersnatch}.
  \end{description}
\end{description}

\subsection{Functions}

\begin{description}
%  \item[$\Gamma$] Unused. %	Γ
  \item[$\Delta$] The accumulation function; certain subscripts are used to denote helper functions: %	Δ
  \begin{description}
    \item[$\Delta_1$] The single-step accumulation function.
    \item[$\Delta_*$] The parallel accumulation function.
    \item[$\Delta_+$] The full sequential accumulation function.
  \end{description}
%  \item[$\gascounter$] Unused. %	Θ
%  \item[$\Pi$] Unused. %	Π
  \item[$\Lambda$] The historical lookup function. See equation \ref{eq:historicallookup}. % Λ
  \item[$\Xi$] The work-digest computation function. See equation \ref{eq:workdigestfunction}. %	Ξ
  \item[$\Upsilon$] The general state transition function. See equations \ref{eq:statetransition}, \ref{eq:transitionfunctioncomposition}. %	ϒ
%  \item[$\Sigma$] Unused. %	Σ
  \item[$\Phi$] The key-nullifier function. See equation \ref{eq:blacklistfilter}. %	Φ
  \item[$\Psi$] The whole-program \textsc{pvm} machine state-transition function. See equation \ref{sec:virtualmachine}. %	Ψ
  \begin{description}
    \item[$\Psi_1$] The single-step (\textsc{pvm}) machine state-transition function. See appendix \ref{sec:virtualmachine}. %	Ψ
    \item[$\Psi_A$] The Accumulate \textsc{pvm} invocation function. See appendix \ref{sec:virtualmachineinvocations}.
    \item[$\Psi_H$] The host-function invocation (\textsc{pvm}) with host-function marshalling. See appendix \ref{sec:virtualmachine}. %	Ψ
    \item[$\Psi_I$] The Is-Authorized \textsc{pvm} invocation function. See appendix \ref{sec:virtualmachineinvocations}.
    \item[$\Psi_M$] The marshalling whole-program \textsc{pvm} machine state-transition function. See appendix \ref{sec:virtualmachine}. %	Ψ
    \item[$\Psi_R$] The Refine \textsc{pvm} invocation function. See appendix \ref{sec:virtualmachineinvocations}.
    \item[$\Psi_T$] The On-Transfer \textsc{pvm} invocation function. See appendix \ref{sec:virtualmachineinvocations}.
  \end{description}
  \item[$\Omega$] Virtual machine host-call functions. See appendix \ref{sec:virtualmachineinvocations}. %	Ω
  \begin{description}
    \item[$\Omega_A$] Assign-core host-call.
    \item[$\Omega_B$] Empower-service host-call.
    \item[$\Omega_C$] Checkpoint host-call.
    \item[$\Omega_D$] Designate-validators host-call.
    \item[$\Omega_E$] Export segment host-call.
    \item[$\Omega_F$] Forget-preimage host-call.
    \item[$\Omega_G$] Gas-remaining host-call.
    \item[$\Omega_H$] Historical-lookup-preimage host-call.
    \item[$\Omega_I$] Information-on-service host-call.
    \item[$\Omega_J$] Eject-service host-call.
    \item[$\Omega_K$] Kickoff-\textsc{pvm} host-call.
    \item[$\Omega_L$] Lookup-preimage host-call.
    \item[$\Omega_M$] Make-\textsc{pvm} host-call.
    \item[$\Omega_N$] New-service host-call.
    \item[$\Omega_O$] Poke-\textsc{pvm} host-call.
    \item[$\Omega_P$] Peek-\textsc{pvm} host-call.
    \item[$\Omega_Q$] Query-preimage host-call.
    \item[$\Omega_R$] Read-storage host-call.
    \item[$\Omega_S$] Solicit-preimage host-call.
    \item[$\Omega_T$] Transfer host-call.
    \item[$\Omega_U$] Upgrade-service host-call.
    \item[$\Omega_W$] Write-storage host-call.
    \item[$\Omega_X$] Expunge-\textsc{pvm} host-call.
    \item[$\Omega_Y$] Fetch data host-call.
    \item[$\Omega_Z$] Pages inner-\textsc{pvm} memory host-call.
    \item[$\Omega_\Taurus$] Yield accumulation trie result host-call.
    \item[$\Omega_\Aries$] Provide preimage host-call.
  \end{description}
\end{description}

\subsection{Utilities, Externalities and Standard Functions}

\begin{description}
  \item[$\mathcal{A}(\dots)$] The Merkle mountain range append function. See equation \ref{eq:mmrappend}.
  \item[$\mathcal{B}_n(\dots)$] The octets-to-bits function for $n$ octets. Superscripted ${}^{-1}$ to denote the inverse. See equation \ref{eq:bitsfunc}.
  \item[$\mathcal{C}(\dots)$] The group of erasure-coding functions.
  \item[$\mathcal{C}_n(\dots)$] The erasure-coding functions for $n$ chunks. See equation \ref{eq:erasurecoding}.
%  \item[$\mathcal{D}$] Unused.
  \item[$\se(\dots)$] The octet-sequence encode function. Superscripted ${}^{-1}$ to denote the inverse. See appendix \ref{sec:serialization}.
  \item[$\mathcal{F}(\dots)$] The Fisher-Yates shuffle function. See equation \ref{eq:suffle}.
  \item[$\hash(\dots)$] The Blake 2b 256-bit hash function. See section \ref{sec:cryptography}.
  \item[$\hash_K(\dots)$] The Keccak 256-bit hash function. See section \ref{sec:cryptography}.
  %  \item[$\mathcal{G}$] Unused.
  %  \item[$\mathcal{I}$] Unused.
  \item[$\mathcal{J}_x$] The justification path to a specific $2^x$ size page of a constant-depth Merkle tree. See equation \ref{eq:constantdepthsubtreemerklejust}.
  \item[$\mathcal{K}(\dots)$] The domain, or set of keys, of a dictionary. See section \ref{sec:dictionaries}.
  \item[$\mathcal{L}_x$] The $2^x$ size page function for a constant-depth Merkle tree. See equation \ref{eq:constantdepthsubtreemerkleleafpage}.
  \item[$\mathcal{M}(\dots)$] The constant-depth binary Merklization function. See appendix \ref{sec:merklization}.
  \item[$\mathcal{M}_B(\dots)$] The well-balanced binary Merklization function. See appendix \ref{sec:merklization}.
  \item[$\mathcal{M}_\sigma(\dots)$] The state Merklization function. See appendix \ref{sec:statemerklization}.
  \item[$\mathcal{N}(\dots)$] The erasure-coding chunks function. See appendix \ref{sec:erasurecoding}.
  \item[$\mathcal{O}(\dots)$] The Bandersnatch ring root function. See section \ref{sec:cryptography} and appendix \ref{sec:bandersnatch}.
  \item[$\mathcal{P}_n(\dots)$] The octet-array zero-padding function. See equation \ref{eq:zeropadding}.
  \item[$\mathcal{Q}(\dots)$] The numeric-sequence-from-hash function. See equation \ref{eq:sequencefromhash}.
  \item[$\mathcal{R}$] The group of erasure-coding piece-recovery functions.
  \item[$\mathcal{S}_k(\dots)$] The general signature function. See section \ref{sec:cryptography}.
  \item[$\mathcal{T}$] The current time expressed in seconds after the start of the \Jam Common Era. See section \ref{sec:commonera}.
  \item[$\mathcal{U}(\dots)$] The substitute-if-nothing function. See equation \ref{eq:substituteifnothing}.
  \item[$\mathcal{V}(\dots)$] The range, or set of values, of a dictionary or sequence. See section \ref{sec:dictionaries}.
  \item[$\mathcal{X}_n(\dots)$] The signed-extension function for a value in $\N_{2^{8n}}$. See equation \ref{eq:signedextension}.
  \item[$\mathcal{Y}(\dots)$] The alias/output/entropy function of a Bandersnatch \textsc{vrf} signature/proof. See section \ref{sec:cryptography} and appendix \ref{sec:bandersnatch}.
  \item[$\mathcal{Z}_n(\dots)$] The into-signed function for a value in $\N_{2^{8n}}$. Superscripted with ${}^{-1}$ to denote the inverse. See equation \ref{eq:signedfunc}.
  \item[$\powset{\dots}$] Power set function.
\end{description}

\subsection{Values}

\subsubsection{Block-context Terms}

These terms are all contextualized to a single block. They may be superscripted with some other term to alter the context and reference some other block.
\begin{description}
  \item[$\mathbf{A}$] The ancestor set of the block. See equation \ref{eq:ancestors}.
  \item[$\mathbf{B}$] The block. See equation \ref{eq:block}.
  \item[$\mathbf{E}$] The block extrinsic. See equation \ref{eq:extrinsic}.
  \item[$\mathbf{F}_v$] The \textsc{Beefy} signed commitment of validator $v$. See equation \ref{eq:beefysignedcommitment}.
  \item[$\mathbf{G}$] The mapping from cores to guarantor keys. See section \ref{sec:coresandvalidators}.
  \item[$\mathbf{G^*}$] The mapping from cores to guarantor keys for the previous rotation. See section \ref{sec:coresandvalidators}.
  \item[$\mathbf{H}$] The block header. See equation \ref{eq:header}.
  \item[$\mathbf{I}$] The sequence of work-reports which were accumulated this in this block. See equation \ref{eq:accumulatedreports}.
  \item[$\mathbf{Q}$] The selection of ready work-reports which a validator determined they must audit. See equation \ref{eq:auditselection}.
  \item[$\mathbf{R}$] The set of Ed25519 guarantor keys who made a work-report. See equation \ref{eq:guarantorsig}.
  \item[$\mathbf{S}$] The set of indices of services which have been accumulated (``progressed'') in the block. See equation \ref{eq:servicestoaccumulate}.
  \item[$\mathbf{T}$] The ticketed condition, true if the block was sealed with a ticket signature rather than a fallback. See equations \ref{eq:ticketconditiontrue} and \ref{eq:ticketconditionfalse}.
  \item[$\mathbf{U}$] The audit condition, equal to $\top$ once the block is audited. See section \ref{sec:auditing}.
  \item[$\mathbf{V}$] The set of verdicts in the present block. See equation \ref{eq:verdicts}.\\
  \item[$\mathbf{W}$] The sequence of work-reports which have now become available and ready for accumulation. See equation \ref{eq:availableworkreports}.
  \item[$\mathbf{X}$] The sequence of transfers implied by the block's accumulations. See equation \ref{eq:deferredtransfers}.\\
\end{description}

Without any superscript, the block is assumed to the block being imported or, if no block is being imported, the head of the best chain (see section \ref{sec:bestchain}). Explicit block-contextualizing superscripts include:
\begin{description}
  \item[$\mathbf{B}^\natural$] The latest finalized block. See equation \ref{sec:bestchain}.
  \item[$\mathbf{B}^\flat$] The block at the head of the best chain. See equation \ref{sec:bestchain}.
\end{description}

\subsubsection{State components}

Here, the prime annotation indicates posterior state. Individual components may be identified with a letter subscript.
\begin{description}
  \item[$\alpha$] The core $\alpha$uthorizations pool. See equation \ref{eq:authstatecomposition}. %	α
  \item[$\beta$] Log of recent activity. See equation \ref{eq:recenthistory}. %	β
  \begin{description}
    \item[$\recenthistory$] Information on the most recent blocks.
    \item[$\beefybelt$] The Merkle mountain belt for accumulating Accumulation outputs.
  \end{description}
  \item[$\gamma$] State concerning Safrole. See equation \ref{eq:consensusstatecomposition}. %	γ
  \begin{description}
    \item[$\gamma_\mathbf{a}$] The sealing lottery ticket accumulator.
    \item[$\gamma_\mathbf{k}$] The keys for the validators of the next epoch, equivalent to those keys which constitute $\gamma_z$.
    \item[$\gamma_\mathbf{s}$] The sealing-key sequence of the current epoch.
    \item[$\gamma_z$] The Bandersnatch root for the current epoch's ticket submissions.
  \end{description}
  \item[$\accountspre$] The (prior) state of the service accounts. %	δ
  \begin{description}
    \item[$\accountspostacc$] The post-accumulation, pre-transfer intermediate state. %	δ
    \item[$\accountspostxfer$] The post-transfer, pre-preimage integration intermediate state. %	δ
  \end{description}
  \item[$\eta$] The e$\eta$tropy accumulator and epochal ra$\eta$domness. %	η
  \item[$\iota$] The validator keys and metadata to be drawn from next. %	ι
  \item[$\kappa$] The validator $\kappa$eys and metadata currently active. %	κ
  \item[$\lambda$] The validator keys and metadata which were active in the prior epoch. %	λ
  \item[$\rho$] The $\rho$ending reports, per core, which are being made available prior to accumulation. %	ρ
  \begin{description}
    \item[$\rho^\dagger$] The post-judgment, pre-guarantees-extrinsic intermediate state. %	δ
    \item[$\rho^\ddagger$] The post-guarantees-extrinsic, pre-assurances-extrinsic, intermediate state. %	δ
  \end{description}
  \item[$\sigma$] The $\sigma$verall state of the system. See equations \ref{eq:statetransition}, \ref{eq:statecomposition}. %	σ
  \item[$\tau$] The most recent block's $\tau$imeslot. %	τ
  \item[$\varphi$] The authorization queue. %	ϕ
  \item[$\psi$] Past judgments on work-reports and validators. %	ψ
  \begin{description}
    \item[$\badset$] Work-reports judged to be incorrect.
    \item[$\goodset$] Work-reports judged to be correct.
    \item[$\wonkyset$] Work-reports whose validity is judged to be unknowable.
    \item[$\offenders$] Validators who made a judgment found to be incorrect.
  \end{description}
  \item[$\chi$] The privileged service indices. %	χ
  \begin{description}
    \item[$\manager$] The index of the blessed service.
    \item[$\delegator$] The index of the designate service.
    \item[$\assigners$] The indices of the services able to assign each core's authorizer queue.
    \item[$\alwaysaccers$] The always-accumulate service indices and their basic gas allowance.
  \end{description}
  \item[$\pi$] The activity statistics for the validators. % π
  \item[$\vartheta$] The accumulation queue. % ϑ
  \item[$\xi$] The accumulation history. % ξ
  \item[$\lastaccout$] The most recent Accumulation outputs. See equation \ref{eq:lastaccout}. % θ
\end{description}

\subsubsection{Virtual Machine components}

\begin{description}
  \item[$\varepsilon$] The exit-reason resulting from all machine state transitions. %	ε
  \item[$\nu$] The immediate values of an instruction. %	ν
  \item[$\memory$] The memory sequence; a member of the set $\mathbb{M}$. %	μ
  \item[$\gascounter$] The gas counter. % ϱ
  \item[$\registers$] The registers. % ω
  \item[$\zeta$] The instruction sequence. %	ζ
  \item[$\varpi$] The sequence of basic blocks of the program. % ϖ
  \item[$\imath$] The instruction counter. % ı
\end{description}

\subsubsection{Constants}

\begin{description}
  \item[$\mathsf{A} = 8$] The period, in seconds, between audit tranches.
  \item[$\mathsf{B}_I = 10$] The additional minimum balance required per item of elective service state.
  \item[$\mathsf{B}_L = 1$] The additional minimum balance required per octet of elective service state.
  \item[$\mathsf{B}_S = 100$] The basic minimum balance which all services require.
  \item[$\mathsf{C} = 341$] The total number of cores.
  \item[$\mathsf{D} = 19,200$] The period in timeslots after which an unreferenced preimage may be expunged.
  \item[$\mathsf{E} = 600$] The length of an epoch in timeslots.
  \item[$\mathsf{F} = 2$] The audit bias factor, the expected number of additional validators who will audit a work-report in the following tranche for each no-show in the previous.
  \item[$\mathsf{G}_A = 10,000,000$] The gas allocated to invoke a work-report's Accumulation logic.
  \item[$\mathsf{G}_I = 50,000,000$] The gas allocated to invoke a work-package's Is-Authorized logic.
  \item[$\mathsf{G}_R = 5,000,000,000$] The gas allocated to invoke a work-package's Refine logic.
  \item[$\mathsf{G}_T = 3,500,000,000$] The total gas allocated across for all Accumulation. Should be no smaller than $\mathsf{G}_A\cdot\mathsf{C} + \sum_{g \in \mathcal{V}(\alwaysaccers)}(g)$.
  \item[$\mathsf{H} = 8$] The size of recent history, in blocks.
  \item[$\mathsf{I} = 16$] The maximum amount of work items in a package.
  \item[$\mathsf{J} = 8$] The maximum sum of dependency items in a work-report.
  \item[$\mathsf{K} = 16$] The maximum number of tickets which may be submitted in a single extrinsic.
  \item[$\mathsf{L} = 14,400$] The maximum age in timeslots of the lookup anchor.
  \item[$\mathsf{N} = 2$] The number of ticket entries per validator.
  \item[$\mathsf{O} = 8$] The maximum number of items in the authorizations pool.
  \item[$\mathsf{P} = 6$] The slot period, in seconds.
  \item[$\mathsf{Q} = 80$] The number of items in the authorizations queue.
  \item[$\mathsf{R} = 10$] The rotation period of validator-core assignments, in timeslots.
  \item[$\mathsf{T} = 128$] The maximum number of extrinsics in a work-package.
  \item[$\mathsf{U} = 5$] The period in timeslots after which reported but unavailable work may be replaced.
  \item[$\mathsf{V} = 1023$] The total number of validators.
  \item[$\mathsf{W}_A = 64,000$] The maximum size of is-authorized code in octets.
  \item[$\mathsf{W}_B = 13,794,305$] The maximum size of an encoded work-package together with its extrinsic data and import implications, in octets.
  \item[$\mathsf{W}_C = 4,000,000$] The maximum size of service code in octets.
  \item[$\mathsf{W}_E = 684$] The basic size of erasure-coded pieces in octets. See equation \ref{eq:erasurecoding}.
  \item[$\mathsf{W}_G = \mathsf{W}_P\mathsf{W}_E = 4104$] The size of a segment in octets.
  \item[$\mathsf{W}_M = 3,072$] The maximum number of imports in a work-package.
  \item[$\mathsf{W}_P = 6$] The number of erasure-coded pieces in a segment.
  \item[$\mathsf{W}_R = 48\cdot2^{10}$] The maximum total size of all unbounded blobs in a work-report, in octets.
  \item[$\mathsf{W}_T = 128$] The size of a transfer memo in octets.
  \item[$\mathsf{W}_X = 3,072$] The maximum number of exports in a work-package.
  \item[$\mathsf{X}$] Context strings, see below.
  \item[$\mathsf{Y} = 500$] The number of slots into an epoch at which ticket-submission ends.
  \item[$\mathsf{Z}_A = 2$] The \textsc{pvm} dynamic address alignment factor. See equation \ref{eq:jumptablealignment}.
  \item[$\mathsf{Z}_I = 2^{24}$] The standard \textsc{pvm} program initialization input data size. See equation \ref{sec:standardprograminit}.
  \item[$\mathsf{Z}_P = 2^{12}$] The \textsc{pvm} memory page size. See equation \ref{eq:pvmmemory}.
  \item[$\mathsf{Z}_Z = 2^{16}$] The standard \textsc{pvm} program initialization zone size. See section \ref{sec:standardprograminit}.
\end{description}

\subsubsection{Signing Contexts}

\begin{description}
  \item[$\mathsf{X}_A = \token{\$jam\_available}$] \emph{Ed25519} Availability assurances.
  \item[$\mathsf{X}_B = \token{\$jam\_beefy}$] \emph{\textsc{bls}} Accumulate-result-root-\textsc{mmr} commitment.
  \item[$\mathsf{X}_E = \token{\$jam\_entropy}$] On-chain entropy generation.
  \item[$\mathsf{X}_F = \token{\$jam\_fallback\_seal}$] \emph{Bandersnatch} Fallback block seal.
  \item[$\mathsf{X}_G = \token{\$jam\_guarantee}$] \emph{Ed25519} Guarantee statements.
  \item[$\mathsf{X}_I = \token{\$jam\_announce}$] \emph{Ed25519} Audit announcement statements.
  \item[$\mathsf{X}_T = \token{\$jam\_ticket\_seal}$] \emph{Bandersnatch Ring\textsc{vrf}} Ticket generation and regular block seal.
  \item[$\mathsf{X}_U = \token{\$jam\_audit}$] \emph{Bandersnatch} Audit selection entropy.
  \item[$\mathsf{X}_\top = \token{\$jam\_valid}$] \emph{Ed25519} Judgments for valid work-reports.
  \item[$\mathsf{X}_\bot = \token{\$jam\_invalid}$] \emph{Ed25519} Judgments for invalid work-reports.
\end{description}
