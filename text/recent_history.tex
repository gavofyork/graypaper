\section{Recent History}\label{sec:recenthistory}

We retain in state information on the most recent $\Crecenthistorylen$ blocks. This is used to preclude the possibility of duplicate or out of date work-reports from being submitted.
\begin{align}
  \label{eq:recentspec}
  \recent &\equiv \tup{\recenthistory, \accoutbelt}\\
  \label{eq:recenthistoryspec}
  \recenthistory &\in \sequence[:\Crecenthistorylen]{\tuple{
    \isa{\rh¬headerhash}{\hash},
    \isa{\rh¬stateroot}{\hash},
    \isa{\rh¬accoutlogsuperpeak}{\hash},
    \isa{\rh¬reportedpackagehashes}{\dictionary{\hash}{\hash}}
  }}\\
  \label{eq:accoutbeltspec}
  \accoutbelt &\in \sequence{\optional{\hash}} \\
  \label{eq:lastaccoutspec}
  \lastaccout &\in \sequence{\tup{\serviceid, \hash}}
\end{align}

For each recent block, we retain its header hash, its state root, its accumulation-result \textsc{mmb} and the corresponding work-package hashes of each item reported (which is no more than the total number of cores, $\Ccorecount = 341$).

During the accumulation stage, a value with the partial transition of this state is provided which contains the correction for the newly-known state-root of the parent block:
\begin{equation}\label{eq:correctlaststateroot}
  \recenthistorypostparentstaterootupdate \equiv \recenthistory\quad\exc\quad\recenthistorypostparentstaterootupdate\subb{\len{\recenthistory} - 1}_\rh¬stateroot = \H_\¬priorstateroot
\end{equation}

We define the new Accumulation Output Log $\accoutbelt$. This is formed from the block's accumulation-output sequence $\lastaccout'$ (defined in section \ref{sec:accumulation}), taking its root using the basic binary Merklization function ($\fnmerklizewb$, defined in appendix \ref{sec:merklization}) and appending it to the previous log value with the \textsc{mmb} append function (defined in appendix \ref{sec:mmr}). Throughout, the Keccak hash function is used to maximize compatibility with legacy systems:
\begin{align}
  \using \mathbf{s} &= \sq{\build{\encode[4]{s} \concat \encode{h}}{\tup{s, h} \orderedin \lastaccout'}}\\
  \label{eq:accoutbeltdef}
  \accoutbelt' &\equiv \mmrappend{\accoutbelt, \merklizewb{\mathbf{s}, \fnkeccak}, \fnkeccak}
\end{align}

The final state transition for $\recenthistory$ appends a new item including the new block's header hash, a Merkle commitment to the block's Accumulation Output Log and the set of work-reports made into it (for which we use the guarantees extrinsic, $\xtguarantees$). Formally:
\begin{equation}
  \begin{aligned}
    \recenthistory' &\equiv {\overleftarrow{\recenthistorypostparentstaterootupdate \append \tup{
      \rh¬reportedpackagehashes,
      \is{\rh¬headerhash}{\blake{\theheader}},
      \is{\rh¬stateroot}{\zerohash},
      \is{\rh¬accoutlogsuperpeak}{\mmrsuperpeak{\accoutbelt'}}
      }}}^\Crecenthistorylen \\
    \where \rh¬reportedpackagehashes &= \set{\build{
        \kv{
          ((g_\xg¬workreport)_\wr¬avspec)_\as¬packagehash
        }{
          ((g_\xg¬workreport)_\wr¬avspec)_\as¬segroot
        }
      }{
        g \in \xtguarantees
      }}
  \end{aligned}
\end{equation}

The new state-trie root is the zero hash, $\zerohash$, which is inaccurate but safe since $\recent'$ is not utilized except to define the next block's $\recentpostparentstaterootupdate$, which contains a corrected value for this, as per equation \ref{eq:correctlaststateroot}.
