\section{Disputes, Verdicts and Judgements}\label{sec:disputes}

\newcommand*{\verdicts}{\mathbf{V}}
\newcommand*{\xtverdicts}{\mathbf{v}}
\newcommand*{\xtculprits}{\mathbf{c}}
\newcommand*{\xtfaults}{\mathbf{f}}

\Jam provides a means of recording \emph{judgements}: consequential votes amongst most of the validators over the validity of a \emph{work-report} (a unit of work done within \Jam, see section \ref{sec:reporting}). Such collections of judgements are known as \emph{verdicts}. \Jam also provides a means of registering \emph{offenses}, judgements and guarantees which dissent with an established \emph{verdict}. Together these form the \emph{disputes} system.

The registration of a verdict is not expected to happen very often in practice, however it is an important security backstop for removing and banning invalid work-reports from the processing pipeline as well as removing troublesome keys from the validator set where there is consensus over their malfunction. It also helps coordinate nodes to revert chain-extensions containing invalid work-reports and provides a convenient means of aggregating all offending validators for punishment in a higher-level system.

Judgement statements come about naturally as part of the auditing process and are expected to be positive, further affirming the guarantors' assertion that the work-report is valid. In the event of a negative judgement, then all validators audit said work-report and we assume a verdict will be reached. Auditing and guaranteeing are off-chain processes properly described in sections \ref{sec:workpackagesandworkreports} and \ref{sec:auditing}.

A judgement against a report implies that the chain is already reverted to some point prior to the accumulation of said report, usually forking at the block immediately prior to that at which accumulation happened. The specific strategy for chain selection is described fully in section \ref{sec:bestchain}. Authoring a block with a non-positive verdict has the effect of cancelling its imminent accumulation, as can be seen in equation \ref{eq:removenonpositive}.

Registering a verdict also has the effect of placing a permanent record of the event on-chain and allowing any offending keys to be placed on-chain both immediately or in forthcoming blocks, again for permanent record.

Having a persistent on-chain record of misbehavior is helpful in a number of ways. It provides a very simple means of recognizing the circumstances under which action against a validator must be taken by any higher-level validator-selection logic. Should \Jam be used for a public network such as \emph{Polkadot}, this would imply the slashing of the offending validator's stake on the staking parachain.

As mentioned, recording reports found to have a high confidence of invalidity is important to ensure that said reports are not allowed to be resubmitted. Conversely, recording reports found to be valid ensures that additional disputes cannot be raised in the future of the chain.

\subsection{The State}

The \emph{disputes} state includes four items, three of which concern verdicts: a good-set ($\goodset$), a bad-set ($\badset$) and a wonky-set ($\wonkyset$) containing the hashes of all work-reports which were respectively judged to be correct, incorrect or that it appears impossible to judge. The fourth item, the punish-set ($\offenders$), is a set of Ed25519 keys representing validators which were found to have misjudged a work-report.
\begin{equation}
  \psi \equiv \tup{\goodset, \badset, \wonkyset, \offenders}
\end{equation}

\subsection{Extrinsic}

The disputes extrinsic, $\xtdisputes$, may contain one or more verdicts $\xtverdicts$ as a compilation of judgements coming from exactly two-thirds plus one of either the active validator set or the previous epoch's validator set, \ie the Ed25519 keys of $\kappa$ or $\lambda$. Additionally, it may contain proofs of the misbehavior of one or more validators, either by guaranteeing a work-report found to be invalid (\emph{culprits}, $\xtculprits$), or by signing a judgement found to be contradiction to a work-report's validity (\emph{faults}, $\xtfaults$). Both are considered a kind of \emph{offense}. Formally:
\begin{equation}
  \begin{aligned}
    \xtdisputes &\equiv (\xtverdicts, \xtculprits, \xtfaults) \\
    \where \xtverdicts &\in \seq{\tup{\H, \ffrac{\tau}{\mathsf{E}} - \N_2, \seq{\tup{\{\top, \bot\}, \N_{\mathsf{V}}, \mathbb{E}}}_{\floor{\nicefrac{2}{3}\mathsf{V}} + 1}}}\\
    \also \xtculprits &\in \seq{\H, \H_E, \mathbb{E}} \,,\quad
    \xtfaults \in \seq{\H, \{\top,\bot\}, \H_E, \mathbb{E}}
  \end{aligned}
\end{equation}

The signatures of all judgements must be valid in terms of one of the two allowed validator key-sets, identified by the verdict's second term which must be either the epoch index of the prior state or one less. Formally:
\begin{align}
  &\begin{aligned}
    &\forall (r, a, \mathbf{j}) \in \xtverdicts, \forall (v, i, s) \in \mathbf{j} : s \in \sig{\mathbf{k}[i]_e}{\mathsf{X}_v \concat r}\\
    &\quad\where \mathbf{k} = \begin{cases}
      \kappa &\when a = \displaystyle \ffrac{\tau}{\mathsf{E}}\\
      \lambda &\otherwise\\
    \end{cases}
  \end{aligned}\\
  &\mathsf{X}_\top \equiv \text{{\small \texttt{\$jam\_valid}}}\,,\ \mathsf{X}_\bot \equiv \text{{\small \texttt{\$jam\_invalid}}}
\end{align}

Offender signatures must be similarly valid and reference work-reports with judgements and may not report keys which are already in the punish-set:
\begin{align}
  \forall (r, k, s) &\in \xtculprits : \bigwedge \left\{
  \begin{aligned}
    &r \in \badset' \,,\\
    &k \in \mathbf{k} \,,\\
    &s \in \sig{k}{\mathsf{X}_G \frown r}
  \end{aligned}\right.\\
  \forall (r, v, k, s) &\in \xtfaults : \bigwedge \left\{ \begin{aligned}
    &r \in \badset' \Leftrightarrow r \not\in \goodset' \Leftrightarrow v \,,\\
    &k \in \mathbf{k} \,,\\
    &s \in \sig{k}{\mathsf{X}_v \concat r}\\
  \end{aligned}\right.\\
  \nonumber\where \mathbf{k} &= \{k_e \mid k \in \lambda \cup \kappa\} \setminus \offenders
\end{align}

Verdicts $\xtverdicts$ must be ordered by report hash. Offender signatures $\xtculprits$ and $\xtfaults$ must each be ordered by the validator's Ed25519 key. There may be no duplicate report hashes within the extrinsic, nor amongst any past reported hashes. Formally:
\begin{align}
  &\xtverdicts = \orderuniqby{r}{\tup{r, a, \mathbf{j}} \in \xtverdicts}\\
  &\xtculprits = \orderuniqby{k}{\tup{r, k, s} \in \xtculprits} \,,\quad
  \xtfaults = \orderuniqby{k}{\tup{r, v, k, s} \in \xtfaults}\\
  &\{r \mid \tup{r, a, \mathbf{j}} \in \xtverdicts\} \disjoint \goodset \cup \badset \cup \wonkyset
\end{align}

The judgements of all verdicts must be ordered by validator index and there may be no duplicates:
\begin{equation}
  \forall (r, a, \mathbf{j}) \in \xtverdicts : \mathbf{j} = \orderuniqby{i}{\tup{v, i, s} \in \mathbf{j}}
\end{equation}


We define $\verdicts$ to derive from the sequence of verdicts introduced in the block's extrinsic, containing only the report hash and the sum of positive judgements. We require this total to be either exactly two-thirds-plus-one, zero or one-third of the validator set indicating, respectively, that the report is good, that it's bad, or that it's wonky.\footnote{This requirement may seem somewhat arbitrary, but these happen to be the decision thresholds for our three possible actions and are acceptable since the security assumptions include the requirement that at least two-thirds-plus-one validators are live (\cite{cryptoeprint:2024/961} discusses the security implications in depth).} Formally:
\begin{align}
  \verdicts &\in \seq{\tup{\H, \{0, \floor{\nicefrac{1}{3}\mathsf{V}}, \floor{\nicefrac{2}{3}\mathsf{V}} + 1\}}} \\
  \verdicts &= \sq{\tup{r, \sum_{\tup{v, i, s} \in \mathbf{j}}\!\!\!\! v}\ \middle\mid\ \tup{r, a, \mathbf{j}} \orderedin \xtverdicts}
\end{align}

There are some constraints placed on the composition of this extrinsic: any verdict containing solely valid judgements implies the same report having at least one valid entry in the faults sequence $\xtfaults$. Any verdict containing solely invalid judgements implies the same report having at least two valid entries in the culprits sequence $\xtculprits$. Formally:
\begin{align}
  \forall (r, \floor{\nicefrac{2}{3}\mathsf{V}} + 1) \in \verdicts &: \exists (r, \dots) \in \xtfaults \\
  \forall (r, 0) \in \verdicts &: |\{(r, \dots) \in \xtculprits\}| \ge 2
\end{align}

We clear any work-reports which we judged as uncertain or invalid from their core:
\begin{equation}\label{eq:removenonpositive}
  \forall c \in \N_\mathsf{C} : \rho^\dagger[c] = \begin{cases}
    \none &\when \{ (\rho[c]_r, t) \in \verdicts, t < \floor{\nicefrac{2}{3}\mathsf{V}} \} \\
    \rho_c &\otherwise
  \end{cases}
\end{equation}

The state's good-set, bad-set and wonky-set assimilate the hashes of the reports from each verdict. Finally, the punish-set accumulates the keys of any validators who have been found guilty of offending. Formally:
\begin{align}
  \goodset' &\equiv \goodset \cup \{r \mid \tup{r, \floor{\nicefrac{2}{3}\mathsf{V}} + 1} \in \verdicts \} \\
  \badset' &\equiv \badset \cup \{r \mid \tup{r, 0} \in \verdicts \} \\
  \wonkyset' &\equiv \wonkyset \cup \{r \mid \tup{r, \floor{\nicefrac{1}{3}\mathsf{V}}} \in \verdicts \} \\
  \offenders' &\equiv \offenders \cup \{ k \mid (r, k, s) \in \xtculprits \}\cup \{ k \mid (r, v, k, s) \in \xtfaults \}
\end{align}

\subsection{Header}\label{sec:judgementmarker}

The verdicts and offenders markers must contain exactly the sequence of report hashes of all new bad \& wonky verdicts and keys of all new offenders, respectively. Formally:
\begin{align}
  \mathbf{H}_j &\equiv \sq{r \mid \tup{r, t} \orderedin \verdicts, t \ne \floor{\nicefrac{2}{3}\mathsf{V}} + 1}\\
  \mathbf{H}_o &\equiv \sq{ k \mid (r, k, s) \in \xtculprits} \frown \sq{ k \mid (r, v, k, s) \in \xtfaults }
\end{align}
